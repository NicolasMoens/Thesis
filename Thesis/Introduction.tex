\section{Introduction}
fqdkfjqmldjf qojfoqi,oqiqoj oqfcqcmoq oqmi qzmoqmo

\subsection{Scientific Rationale}
In various systems it is important to take into account a radiation field when considering gas dynamics. Radiation can exert forces on the gass and transport energy. An important concept in these sort of systems is the Eddington factor $\Gamma$, this is the ratio between the radiative and gravitational acceleration.
\begin{align}
	\Gamma = \frac{\kappa F}{c g_{grav}}
\end{align}
Where $F$ is the radiation energy flux, $\kappa$ the absorption opacity, $c$ the speed of light in vacuum and $g_{grav}$ the magnitude of the local gravitational acceleration. Systems that ecceed the Eddington limit ($\Gamma > 1$) are called super-Eddington systems. A prime example for such a system is $\eta$ Carinae.\\

Other astrophysical regimes where radiation plays an important role are for example stellar winds, accretion discs, very massive stars, ...

\subsection{Hydrodynamics}
The movement of non-isothermal gasses and fluids, in the absence of magnetic fields and radiation, are described by hydrodynamics. The hydrodynamical equations describe conservation of mass, conservation of momentum and conservation of energy in the following equations:

\begin{align}
 \partial_t \left(\rho \right) + \vec{\nabla} \cdot \left( \rho \vec{v}  \right) &= S_\rho \\
 \partial_t \left(\rho \vec{v} \right) + \vec{\nabla} \cdot \left( \vec{v} \rho \vec{v} + p \right)    &= S_{\rho \vec{v}} \\
 \partial_t \left(e \right) + \vec{\nabla} \cdot \left( \vec{v} e + \vec{v} p \right) &= S_e
\end{align}

These three partial differential equations are called the continuity equation (describing mass density), the momentum equation (describing momentum density) and the energy equation (describing energy density). Above, the equations were written in their conservative form, they al have the same shape:

\begin{align}
	\partial_t \underbrace{u}_\text{density} + \vec{\nabla} \cdot \overbrace{\vec{F_u}}^\text{density flux} &= \underbrace{S_u}_\text{sourceterm}
\end{align}

There are only 3 PDE's describing 4 primitive variables: $\rho$, $\vec{v}$, $e$ and $p$. This means that there is need for an additional closure relation:
\begin{align}
	p = (\gamma - 1) \left(e - \frac{\rho \vec{v}^2}{2} \right)
\end{align}

The sourceterms in the equations describe adding or substracting to one of the densities: $S_\rho$ describes mass being added or substracted from the system, $S_{\rho \vec{v}}$ describes external forces such as gravity or radiative forces and $S_e$ describes work excerted on the system, this can be for example due to heating of the fluid.\\

Predicting fluid dynamics means solving these forementioned equations simultaniously, this is done using computercodes such as mpi-amrvac CITE HERE. The Evolution of a fluid will depend heavily on both initial and boundary conditions.\\


\subsection{Radiation Hydrodynamics}
The main governing equation describing the evolution of radiation trough a medium is the radiative transfer equation:
\begin{align}
\frac{1}{c} \frac{\partial I_\nu}{\partial t} = j_\nu + \int_\Omega \frac{\kappa_\nu}{4 \pi} I_\nu d\Omega
\end{align}
Where $I_\nu$ is the intensity at frequency $\nu$, $j_\nu$ is the mean intensity and $\int_\Omega d\Omega$ is an integration over the total solid angle.


\subsection{Sobolev and CAK-theory}

\subsection{Flux Limited Diffusion}
\subsubsection{Elliptic vs Parabolic}